% Encodage des caractères et langue du document
\usepackage{setspace}
\usepackage[T1]{fontenc}
\usepackage[utf8]{inputenc}
\usepackage{lmodern}
\usepackage[english,french]{babel}
\usepackage{textgreek}

% Keep proof headings in English even when French is the active babel language.
% (babel's French captions set \proofname to "Démonstration" by default.)
\addto\extrasfrench{\renewcommand{\proofname}{Proof}}

\setcounter{tocdepth}{3} % Pour que les subsubsections n'apparaissent pas dans la TOC
\setcounter{secnumdepth}{3} % Pour que les subsubsections ne soient pas numérotées
\usepackage{fixltx2e}

%%%%%%%%%% Gestions des marges %%%%%%%%%% 
\usepackage{geometry} % Si on a besoin d'une configuration plus précise des marges
\geometry{a4paper,                % format de papier
% Définition des marges :
  left= 2.5cm,right = 2.5cm,  % marge identique des deux côtés
  top = 3cm,bottom = 3cm,
% En-tête et pied de page :
  headheight=6mm,         % espace réservé à l'en-tête dans la marge top
  %headsep=3mm,            % espace entre le corps et l'en-tête
  %footskip=9mm            % espace entre le corps et le pied de page
  marginparwidth = 16mm,
  asymmetric              % toutes les pages ont les mêmes marges
}

\raggedbottom
\reversemarginpar
% \usepackage{showframe} % pour afficher les traits des marges

%%%%%%%%%%%%%%%%%%%%%%%%%%%%%%%%%%%%%%%%%%%%%%%%%%%%%%%%%%%%%%%%%%%%
%%%%%%%%%%% Gestion maths %%%%%%%%%% 
\usepackage{amsmath,amssymb,amsfonts,amsthm}
\usepackage{bm} % for bold math symbols
\usepackage{mathtools,empheq} % mathtools (enhanced amsmath) and empheq (boxed equations)
\usepackage{mathrsfs}% pour rajouter un format de lettres façon calligraphie en math mode.
\DeclareMathOperator{\sinc}{sinc}
\DeclareMathOperator{\e}{e}

\usepackage[locale = FR]{siunitx} % Pour gérer les unités
\sisetup{inter-unit-product=\ensuremath{{}\cdot{}}} % pour mettre des points médians entre les unités quand il y en a plusieurs
\sisetup{separate-uncertainty=true,multi-part-units=single} % pour faire des incertitudes en écrivant \SI{valeur(incertitude)}{unité}
\DeclareSIUnit\vitesse{\meter\per\second}
\usepackage{eurosym}
\DeclareSIUnit{\octet}{o}

%%%%%%%%%%%%%%%%%%%%%%%%%%%%%%%%%%%%%%%%%%%%%%%%%%%%%%%%%%%%%%%%%%%%
%%%%%%%%%%%  Graphics / Table / List %%%%%%%%%%%
\usepackage{graphicx,array,tikz,multirow}
% \usepackage{tikz}
\usetikzlibrary{circuits.ee.IEC}
\usepackage[european resistor, european voltage, european current]{circuitikz}
\usetikzlibrary{arrows, shapes, positioning}
\usetikzlibrary{decorations.markings, decorations.pathmorphing,
decorations.pathreplacing}
\usetikzlibrary{calc,patterns,shapes.geometric}
\usetikzlibrary{shapes.multipart, positioning, patterns, backgrounds}
\tikzstyle{titlebox}=[rectangle,inner sep=10pt,inner ysep=10pt,draw]%
\usetikzlibrary{arrows.meta}
\usetikzlibrary{matrix, fit}
\tikzset{global scale/.style={
    scale=#1,
    every node/.append style={scale=#1}
  }
}


\usepackage{caption,subcaption} % permet de faire des subfigures (remplace le package subfig)
\usepackage{svg,float}
\usepackage{booktabs,paralist}
\newcolumntype{x}[1]{>{\centering\arraybackslash\hspace{0pt}}p{#1}}
\usepackage[section]{placeins}
\usepackage{hanging}

%%%%%%%%%%%%%%%%%%%%%%%%%%%%%%%%%%%%%%%%%%%%%%%%%%%%%%%%%%%%%%%%%%%%
%%%%%%%%%%% Header / Foot %%%%%%%%%%% 
\usepackage{fancyhdr,emptypage} % garantit que les pages blanches avant les débuts de chapitres soient vraiment blanches (pas d'en-tête ni de pied de page)
\let\cleardoublepage\clearpage
 
\fancypagestyle{plain}{ %% Page chapitre, toc ...
    \fancyhead{}\fancyfoot[C]{\thepage}
    \renewcommand{\headrulewidth}{0pt}
    \renewcommand{\footrulewidth}{0pt}
}

%%%%%%%%%%% Page normale
\pagestyle{fancy}
    \renewcommand{\chaptermark}[1]{\markboth{\chaptername \ \thechapter.\ #1}{}} % sert à personnaliser l'affichage de \leftmark (ici : le mot "Chapitre", le numéro, un point, et le titre du chapitre, sans écrire en majuscules)
    % \renewcommand{\chaptermark}[1]{\markleft{\chaptername \ \thechapter.\ #1}{}}
    % \renewcommand{\sectionmark}[1]{\markright{\thesection.\ #1}} % sert à personnaliser l'affichage de \rightmark (ici : le numéro et le titre de la section en cours, sans écrire en majuscules)
    \fancyhf{} % assure que les entête et pieds de page sont vides au départ
    \fancyhead[C]{\selectfont\nouppercase{\leftmark}}
    \fancyfoot[C]{\thepage}
% Explications :
% L = left, R = right, C = center, E = even pages, O = odd pages
%\leftmark : adds name and number of the current top-level structure (for example, Chapter for reports and books classes; Section for articles ) in uppercase letters.
%\rightmark : adds name and number of the current next to top-level structure (Section for reports and books; Subsection for articles) in uppercase letters.

%%%%%%%%%%% Personnaliser les premières pages des chapitres
\usepackage[Lenny]{fncychap}
\ChNameVar{\fontsize{25}{25}\usefont{OT1}{phv}{m}{n}\selectfont}
\ChRuleWidth{0pt}
\ChNumVar{\fontsize{60}{62}\selectfont\textcolor{curcolor}}

\makeatletter
\ChTitleVar{\Huge\rm}
\renewcommand{\DOCH}{%
\setlength{\fboxrule}{\RW} % Let fbox lines be controlled by
\fbox{\CNV\FmN{\@chapapp}\space \CNoV\thechapter}\par\nobreak
\vskip 20\p@}
\renewcommand{\DOTIS}[1]{%
\CTV\bfseries\FmTi{#1}\par\nobreak
\vskip 20\p@}
\makeatother

\renewcommand{\thesection}{\arabic{section}}

% Pour la table des matières
\usepackage[francais,nohints,tight]{minitoc}		% Mini table des matières, en français
\setcounter{minitocdepth}{2} % Mini-toc détaillées (sections/sous-sections)
\setlength{\mtcindent}{-1em} % décalage des minitoc à gauche
\dominitoc

\usepackage[nottoc]{tocbibind} % pour que la bibliographie apparaisse dans la table des matières (avec l'option pour que la table des matières elle-même n'apparaisse pas dans la table des matières).
% \usepackage{tocloft}% pour pouvoir modifier les tailles d'espacement dans la table des matières
\usepackage[titles]{tocloft}

%%%%%%%%%%%%%%%%%%%%%%%%%%%%%%%%%%%%%%%%%%%%%%%%%%%%%%%%%%%%%%%%%%%%
%%%%%%%%%%% Divers %%%%%%%%%%% 
\usepackage{textcomp} % rajoute des symboles 
\usepackage{xcolor} % pour ajouter de la couleur (si besoin)
\usepackage{epigraph} % pour rajouter des citations en début de chapitre  \epigraph{Citation}}{Auteur}
\usepackage{titling}
\usepackage{lipsum} 
\usepackage{csquotes} % added 07/09/21

\usepackage{xspace}
\usepackage{afterpage}
\renewcommand{\baselinestretch}{1.2} % interligne

\usepackage[textsize=footnotesize]{todonotes}
\usepackage{comment} % enable \begin{comment} ... \end{comment} blocks
% \graphicspath{{./}{./ch3/}} % allow images referenced as `img/...` in chapter files to resolve to `ch3/img/...` when compiling from project root

%%%%%%%%%%%%%%%%%%%%%%%%%%%%%%%%%%%%%%%%%%%%%%%%%%%%%%%%%%%%%%%%%%%%
%%%%%%%%%%% Links ref  %%%%%%%%%%% 
\usepackage{bookmark}
\usepackage{acronym}
\usepackage[nameinlink,french]{cleveref} % noabbrev
\Crefname{figure}{Fig.}{Figs.} % traduction des références aux figures/tables/équations
\crefname{figure}{fig.}{figs.}
\Crefname{equation}{Eq.}{Eqs.}
\crefname{equation}{eq.}{eqs.}
\Crefname{table}{Table.}{Tables.}
\crefname{table}{table.}{tables.}

% Configuration de hyperref
\definecolor{color_ref}{rgb}{0.18, 0.31, 0.31} % couleur cite
\definecolor{color_link}{RGB}{36, 56, 141}
\definecolor{curcolor}{RGB}{113,127,184} % couleur des liens (bleu clair)

\hypersetup{
	colorlinks=true, % colore les liens au lieu de les encadrer
	pdfstartview=FitV, % ouvre le PDF de façon à ce qu'il prenne la taille verticale de l'écran
	urlcolor=color_link, % choix de la couleur des liens URL
	linkcolor= color_link, % choix de la couleur des liens internes (table des matières, etc.)
	citecolor=color_ref % choix de la couleur des liens de citations
}

%%%%%%%%%%%%%%%%%%%%%%%%%%%%%%%%%%%%%%%%%%%%%%%%%%%%%%%%%%%%%%%%%%%%
%%%%%%%%%%% Bibliography ref  %%%%%%%%%%%
\usepackage[hyperref=true,natbib=true,
			backref=true,date=year,
			backend=biber,
			url=false,doi=false,isbn=false,%
			minbibnames=6, % nb min authors in biblio
			maxbibnames=6, % nb max authors in biblio
			maxcitenames=1,mincitenames=1, % nb min authors as textual
			maxalphanames=1, %nb author ref
			style=alphabetic,%authoryear, numeric ,  alphabetic
			sorting=nyt]{biblatex}
\renewcommand*{\bibfont}{\footnotesize}
\setlength\bibitemsep{\itemsep}
\renewbibmacro{in:}{} % remove In 

\renewcommand*{\labelalphaothers}{}
\DeclareLabelalphaTemplate{
  \labelelement{
    \field[final]{shorthand}
    \field{labelname}
    \field{label}
  }
  \labelelement{\literal{,\addhighpenspace}}
  \labelelement{\field{year}}
}

\AtEveryBibitem{%
    \clearfield{note} % Remove note
    \clearlist{language} % Remove doi
}

%%%% HUY - Some useful math commands
\def\blacksquare{\hbox{\vrule width 4pt height 4pt depth 0pt}}
% Use the standard \square from amssymb to avoid a broken custom definition
\renewcommand{\qed}{\hfill$\square$}
\def\qedp{\rightline{$\blacksquare$}}
\def\EXPT{\mathop{{\rm E}}}
\def\MIN{\mathop{{\rm min}}}
\def\MAX{\mathop{{\rm max}}}
\def\LIM{\mathop{{\longrightarrow}}}
\def\UNION{\mathop{\bigcup}}
\def\INTERSECT{\mathop{\bigcap}}
\def\ARGMIN{\mathop{{\rm argmin}}}
\def\ARGMAX{\mathop{{\rm argmax}}}
\def\TOUND{\mathop{\longrightarrow}}
\def\col#1{\mathop{{\rm col}\,\left(\,#1\,\right)}}
\def\eqref#1{(\ref{#1})}
\def\rank{\mathop{{\rm rank}}}

% User definition from Shengya
\def\V{\mathcal{V}} % the set of vehicle
\def\E{\mathcal{E}} % the set of edge
\def\G{\mathcal{G}} % graph
\def\A{\mathcal{A}} % adjacency matrix 
\def\W{\mathcal{W}} % weighted adjacency matrix
\def\L{\mathcal{L}} % Laplacian matrix
\def\N{\mathcal{N}} % neighbors
\def\LO{\mathcal{LO}} % the set of observer
\def\DO{\mathcal{DO}} % the set of observer
\def\diag#1{\mathop{{\rm diag}\,\left(\,#1\,\right)}}
\def\ST{\mathcal{ST}} % Self trust
\def\LT{\mathcal{LT}} % Local trust
\def\DT{\mathcal{DT}} % Distributed trust 
\def\VT{\mathcal{VT}} % Vehicle trust 
\def\AT{\mathcal{AT}} % Aggregation trust
\def\T{\mathcal{T}} % Aggregation trust
\def\O{\mathcal{O}} % opinion

% User Defined Color based on RGB
\definecolor{userblue}{RGB}{31,119,180}
\definecolor{userorange}{RGB}{255,127,14}
\definecolor{usergreen}{RGB}{44,160,44}
\definecolor{userpurple}{RGB}{148,103,189}

\definecolor{blue1}{RGB}{31,119,180}    % 车辆/物理层
\definecolor{orange1}{RGB}{255,127,14}  % 本地观测器
\definecolor{green1}{RGB}{44,160,44}    % 信息层/通信
\definecolor{purple1}{RGB}{148,103,189} % 分布式观测器
\definecolor{rose1}{RGB}{227, 119, 194}

\newcommand{\widebox}[2][0.5em]{\fbox{\hspace{#1}$\displaystyle #2$\hspace{#1}}}


%\newcommand*\rank{\rm rank}
% \def\rank{\mathop{{\rm rank}}}

%%%
\newtheorem{prb}{\it \textbf{Problem}}%[section]
\newtheorem{theorem}{\it \textbf{Theorem}}%[section]
\newtheorem{lemma}{\it \textbf{Lemma}}%[section]
\newtheorem{definition}{\it \textbf{Definition}}
\newtheorem{remark}{\it \textbf{Remark}}%[section]
\newtheorem{assumption}{\it \textbf{Assumption}}%[section]
\newtheorem{example}{\it \textbf{Example}}%[section]b
\newtheorem{annex}{\it Annex}%[section]b
\newtheorem{prop}{\it \textbf{Proposition}}%[section]b
\newtheorem{proposition}{\it \textbf{Proposition}}%[section]b

% Package for dashed rules
\usepackage{dashrule}
% Create alias for compatibility
% \newcommand{\dashedhrule}[2]{\hdashrule{#1}{#2}}

\newcommand{\dashedhrule}[3][\linewidth]{%
  \par\noindent\hdashrule[0pt]{#1}{#2}{#3}\par
}

% Custom commands for subscripts used in the document
\newcommand{\refx}{\text{ref}}
\newcommand{\uio}{\text{uio}}
\newcommand{\nn}{\text{nn}}
\newcommand{\Lo}{L}

