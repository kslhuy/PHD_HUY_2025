\usepackage[nomaketitle]{config/ul_cover/ul-cover}

\sujet{Development of new advanced estimation algorithms for improving the resilience of the autonomous and
connected vehicle}
\author{Quang Huy NGUYEN}
\encadrant{Ali ZEMOUCHE \\ Hugh Rafaralahy \\ Madjid HADDAD}

\institute{Université de Lorraine}
\doctoralschool{Titre de l'école doctorale}{xxx}
\specialty{Automatique}
\laboratory{Centre de Recherche Nancy}
\date{xx mois 202x}

\ulassetspath{config/ul_cover}

\jurymember{1}{Prénom \textsc{Nom 1}}{Affiliation & \emph{Examinateur}}{} % le président doit être en premier
\jurymember{2}{Prenom \textsc{Nom 2}}{Affiliation}{Rapporteur}
\jurymember{3}{Prenom \textsc{Nom 3}}{Affiliation}{Rapporteur}
\jurymember{4}{Prenom \textsc{Nom 4}}{Affiliation}{Rapporteur}
\jurymember{5}{Prenom \textsc{Nom 5}}{Affiliation}{Rapporteur}
\jurymember{6}{Prenom \textsc{Nom 6}}{DR Affiliation}{Directeur de thèse}
% \jurymember{7}{Prénom NOM}{Titre, établissement}{Directeur de thèse}

% entre 1700 et 4000 signes ? 
\frabstract{
% \lipsum[10-11]
}

\enabstract{
The reliability and safety of Connected and Autonomous Vehicles (CAVs) critically depend on the real-time availability of precise dynamic state information. However, equipping vehicles with comprehensive high-end sensor suites is often limited by economic and physical constraints. Furthermore, CAVs are increasingly vulnerable to sensor faults and cyber-attacks targeting inter-vehicle communications (V2V/V2I), which pose significant risks to system integrity. To address these challenges without incurring the prohibitive costs of hardware redundancy, this thesis proposes a resilient ``soft sensor'' framework based on analytical redundancy.

This research adopts a hybrid modeling methodology that integrates physical differential equations with online learning-based neuro-adaptive neural networks. This approach enables the accurate representation of complex, time-varying parameters and unmodeled dynamics. Building on these models, advanced nonlinear estimation algorithms—specifically discrete-time Unknown Input Observers (UIO) and neuro-adaptive observers—are developed to reconstruct unmeasured states and ensure system resilience against data loss and malicious signal injection.

The effectiveness of the proposed strategies is demonstrated in two primary control scenarios:
\begin{itemize}
    \item \textbf{Cyber-Secure Cooperative Adaptive Cruise Control (CACC):} Novel observer architectures are designed to detect and mitigate cyber-threats, such as Denial of Service (DoS) and False Data Injection (FDI). These fast-converging discrete-time observers enable the reconstruction of attack signals, ensuring robust longitudinal control.
    \item \textbf{Resilient Platooning and Vehicle Tracking:} A distributed estimation framework, enhanced by a trust management mechanism, is proposed to handle nonlinear dynamics and filter out unreliable data, thereby maintaining platoon cohesion during complex maneuvers like lane changes.
\end{itemize}

Finally, the theoretical contributions are validated through the development of a prototype embedded electronic board. This hardware platform demonstrates a cost-effective and scalable solution for enhancing the autonomy, security, and fault tolerance of next-generation vehicles.
}

\frkeywords{Estimation, mot clé 2, mot clé 3, mot clé 4}
\enkeywords{State Estimation, keyword 2, keyword 3, keyword 4}


%%%%%%% Paramètres PDF
\title{\sujet}
\hypersetup{
pdftitle={Thesis Year},
pdfsubject={\sujet},
pdfauthor={Quang Huy NGUYEN},
pdfkeywords={Estimation, mot clé 2, mot clé 3, mot clé 4},
}
