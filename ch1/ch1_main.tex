\chapter{Introduction générale}\label{chp1_intro}
\objectif{Objectif rapide du chapitre \lipsum[30]}

%%%%%%%%%%%%%%%%%%%%%%%%%%%%%%%%%%%%%%%%%%%%%%%%%%%%%%%%%%%%%%%%%%%%%%%%%%%%%%%%%%%%%%%%%%%%%%%%%
\section{Quelques conseils d'utilisation}
%%%%%%%%%%%%%%%%%%%%%%%%%%%%%%%%%%%%%%%%%%%%%%%%%%%%%%%%%%%%%%%%%%%%%%%%%%%%%%%%%%%%%%%%%%%%%%%%%

\subsection{Organisation du projet}

\paragraph{Fichier maître à compiler}
Le fichier \texttt{main.tex} contient tous les sous-fichiers à inclure. C'est l'organisation du document.

\paragraph{Configuration}
Le dossier \texttt{config/} contient les fichiers de parametrage du document, les macros, ainsi que le package pour la couverture PSL (\cref{pslcover}).

\paragraph{Chapitres}
Ils sont situés dans des dossiers indépendants, avec des dossiers de figures associés (\texttt{.../fig/}). L'auto-complétion permet de trouver les figures associées au chapitre voulu : \texttt{ch1/fig/xxxx.jpeg}.

\paragraph{Autres fichiers et pages}
Les autres fichiers de texte (conclusion, remerciements), et information (info, liste des publications, glossaire) sont situés dans le dossier \texttt{others/}.

\subsection{Couvertures par Pierre Guillou pour PSL}\label{pslcover}
Seules les couvertures sont imposées par PSL. Le fichier ici a été réalisée par \href{https://pierre.guillou.net/psl-cover/2018/}{Pierre Guillou}\footnote{\url{https://pierre.guillou.net/psl-cover/2018/}} et est disponible sur le site dans les \href{https://collegedoctoral.psl.eu/doctorat-psl/espace-ressources/}{ressources du site de PSL}\footnote{\url{https://collegedoctoral.psl.eu/doctorat-psl/espace-ressources/}} (Version 1.2 - 20 juillet 2019).
A noter que le président du jury doit être sur la première ligne de la liste du jury.

\subsection{Illustrations}
Overleaf propose l'auto-completion pour les figures de figures, ce qui est assez rapide et efficace.
Un label est ajouté pour chaque figure afin de la citer après. La figure suivant est référencée par le label \texttt{fig1\_cir} car c'est une figure du chapitre 1. Pour une table du chapitre 1, on pourra utiliser \texttt{tab1\_power}.
La figure [\cref{fig1_circsystem}a] illustre ce propos, pour plus de détails voir le \cref{chp2_chap}.

\begin{figure}[H]\centering
\subcaptionbox{Réseau vasculaire}[.4\linewidth]{\includegraphics[height=4cm]{ch1/fig/Circulatory_System_fr.jpg}}
\subcaptionbox{Réseau capillaire sanguin}[.55\linewidth]{\includegraphics[height=4cm]{ch1/fig/Blood_vessels.png}}
\caption{(a) Le réseau vasculaire est composé des réseaux artériel et veineux. (b) Les capillaires sanguins réalisent la jonction entre les deux réseaux. [Wikimedia]}
\label{fig1_circsystem}
\end{figure}

%%%%%%%%%%%%%%%%%%%%%%%%%%%%%%%%%%%%%%%%%%%%%%%%%%%%%%%%%%%%%%%%%%%%%%%%%%%%%%%%%%%%%%%%%%%%%%%%%
\section{Une deuxième section}

\subsection{Acronymes}
C'est assez pratique d'utiliser des acronymes, mais il faut les définir avant. \todo[inline]{Ici, un commentaire}
Ici on va utiliser le \acrshort{gpu} ou la \acrlong{ram}, ou encore \acrfull{ssd}\todo{A faire}.

%%%%%%%%%%%%%%%%%%%%%%%%%%%%%%%%%%%%%%%%%%%%%%%%%%%%%%%%%%%%%%%%%%%%%%%%%%%%%%%%%
\subsection{Quelques citations}
\subsubsection{Avec la commande Cite}
On peut citer des références \cite{ueda_tissue_2020,thomas_microbubble_2013} pour appuyer le propos.
La table des matières est réalisée avec \texttt{biblatex}.
\subsubsection{En synchronisant Zotero}
Zotero peut être synchronisé pour faire la gestion de la bibliographie. Il ne suffit que de mettre à jour le \texttt{.bib} de temps en temps.
