\chapter{Introduction générale}\label{chp1_intro}
\objectif{This chapter lays the foundation for the thesis by outlining the research context, challenges, and objectives in resilient state estimation for connected autonomous vehicles. It motivates the need for hybrid and cooperative observer designs, summarizes the main contributions, and presents the organization of the manuscript.}

%%%%%%%%%%%%%%%%%%%%%%%%%%%%%%%%%%%%%%%%%%%%%%%%%%%%%%%%%%%%%%%%%%%%%%%%%%%%%%%%%%%%%%%%%%%%%%%%%

\section{Historical Background of Estimation Theory and Observer Design}
State observers have been a cornerstone of control theory since the 1960s. The classical Luenberger observer, introduced by Luenberger (1964/1966), uses available inputs and outputs to reconstruct the internal state of a plant and asymptotically drive the estimate toward the true state. In parallel, Kalman's work on the Kalman filter (1960) established the foundations of stochastic state estimation, widely adopted for navigation and sensor fusion.

As control applications grew in complexity (e.g., aerospace and automotive systems), observer theory expanded to cope with unknown inputs, such as disturbances or unmeasured actuator effects. Unknown Input Observers (UIOs) were developed to estimate states despite these signals, under algebraic design conditions (often expressed as rank/matching constraints), and were applied extensively to fault detection and isolation. Representative contributions include the UIO-based fault-diagnosis frameworks of Hou and M\"uller (1992) and Chen and Patton (1999). In the same period, sliding-mode observers emerged as a robust alternative: by injecting discontinuous corrective terms, they can enforce finite-time error convergence and, in sliding motion, reduce sensitivity to certain matched uncertainties, enabling reconstruction of faults or disturbances via the so-called equivalent input.

Throughout the 1990s and 2000s, a broad family of robust and adaptive designs---including $H_\infty$, high-gain, and adaptive observers---were proposed to maintain estimation performance under modeling errors, noise, and component faults. In automotive systems, these observers and filters became core building blocks, from early ECUs using Kalman filtering to multi-sensor schemes estimating lateral dynamics (e.g., slip angle or roll angle), wheel slip for anti-lock braking, and difficult-to-measure quantities such as road bank angle or tire forces modeled as disturbances. This historical evolution, from classical linear observers to robust and unknown-input extensions, sets the stage for today's resilient estimation frameworks.
\section{Recent Advances in Hybrid and Resilient Estimation}
Contemporary research in estimation builds on this legacy, addressing new challenges posed by connected and autonomous vehicles. One major trend is the development of hybrid observer architectures that combine model-based and data-driven elements. Purely data-driven estimators (e.g. deep neural network filters) can achieve high accuracy by learning complex patterns, but they lack transparency and are difficult to certify for safety-critical use. Conversely, purely model-based observers provide analytical guarantees of stability/robustness but may become inaccurate if the model is incomplete or if classical existence conditions (like observability rank or matching conditions) are violated. To overcome these trade-offs, hybrid approaches integrate physics-informed observer design with learning mechanisms. For example, a nominal observer (such as a Luenberger or Kalman filter or an Unknown Input Observer) is used to ensure basic robustness, while a learning module (e.g. a neural network or adaptive element) concurrently compensates for modeling errors or time-varying parameters. Recent studies have shown the promise of this approach. Jeon et al. (2024) develop a neuro-adaptive state observer where a neural network estimates unmodeled nonlinear dynamics; the design guarantees exponential stability of both the state estimation error and the neural parameter error, with even an $H_\infty$ performance bound when the learned approximation is imperfect. They further propose a switching hybrid estimator that alternates between learning (system identification) mode and a conventional observer mode, enabling online parameter updates only when sufficient excitation is present. This illustrates how blending learning with traditional observers can yield resilient estimators that adapt to unknown dynamics while preserving theoretical guarantees.

Another area of rapid progress is resilient state estimation under cyber-attacks and faults. As vehicles become connected, the estimation algorithms must contend with adversarial inputs (malicious sensor data or spoofed communication) in addition to natural disturbances. Classical robust observers treated unknown disturbances as benign inputs, but modern resilient estimation explicitly models intelligent attacks that may seek to deceive the estimator. A common strategy is to treat attacks as additional unknown inputs or to design observers that are robust to outliers. Unknown Input Observers have been revisited in this context: for instance, Jeon et al. (2020) demonstrated simultaneous cyber-attack detection and sensor health monitoring in a connected vehicle platoon using a bank of observers. In that work, deviations in an observer’s residual signaled either a malicious data injection or a sensor fault, allowing the system to detect and isolate compromised signals. Sliding mode techniques have also been adapted for resilience – their intrinsic robustness to matched uncertainties is naturally suited to detecting attacks that enter through actuators or certain sensors. Recent studies integrate high-order sliding mode differentiators to estimate and reconstruct attack signals in real time. For example, Sadki et al. (2025) use an adaptive higher-order sliding differentiator as part of a cyberattack estimation scheme on a connected vehicle, achieving finite-time convergence of the attack estimate despite measurement noise. Meanwhile, researchers have relaxed the stringent conditions required by earlier unknown-input methods. The classical observer matching condition (which demands that each unknown input affect only certain state subspaces) severely limited applicability, since many practical systems – especially networked vehicles – do not satisfy this constraint. Advanced designs now employ geometric and optimization-based approaches to bypass these limitations. Floquet et al. (2007) and Kalsi et al. (2010) introduced strategies to augment the system outputs (via auxiliary high-gain filters or differentiators) such that a sliding-mode UIO can be constructed even when the original rank condition does not hold. More recently, Zhao et al. (2025) developed a distributed unknown input observer that leverages a joint condition across multiple vehicles in a platoon, instead of per-vehicle conditions, thereby allowing cooperative estimation where each vehicle’s observer overcomes local unobservability by fusing information from neighbors. Such cooperative estimation approaches are especially relevant to connected automated driving, as they enable an entire platoon or fleet of vehicles to collectively estimate critical states and disturbance/attack inputs that no single vehicle could estimate in isolation.

In tandem with observer innovations, the community has explored data-driven estimation enhancements. Learning-based observers encompass methods like neural network observers, Gaussian process filters, and physics-informed neural networks (PINNs) embedded into observers. These techniques aim to improve estimation accuracy for highly nonlinear or partially known systems by learning from data. Crucially, they are often designed in a closed-loop manner with the system model: rather than replacing the model entirely, they correct or augment it. For example, neural network function approximators have been used to estimate vehicle tire-road friction or unknown aerodynamic forces, augmenting a nominal vehicle model observer. By training these networks online (adaptive) or offline, the observer can gradually “learn” the unmodeled dynamics while the model-based portion ensures stability bounds. This hybrid learning paradigm is visible in many recent works on autonomous vehicles, often under names like neural observers, adaptive high-gain observers, or observer-based learning. Across the board, the thrust of current research is clear: marry the reliability of control-theoretic observers with the flexibility of machine learning. The result is estimation algorithms that are more resilient to both unexpected physical disturbances and adversarial attacks, which is vital as vehicles operate in open and uncertain environments.

\subsection{Industrial relevance and deployment challenges}
Bridging these advanced techniques from research to industry is a non-trivial task. Automotive systems are a prime example of cyber-physical systems where estimation algorithms must satisfy real-world constraints beyond theoretical performance. One major consideration is functional safety: industry standards (e.g. ISO 26262) demand that any component affecting vehicle control be robustly verified and fail-safe. This means an estimator must not only perform well nominally, but also handle worst-case scenarios (sensor failures, extreme environmental conditions, communication dropouts) without causing unsafe behavior. Model-based observers have an advantage here – their behavior under faults can often be bounded or analyzed using control theory. For instance, a well-designed $H_\infty$ observer or interval observer can guarantee that estimation errors remain within set bounds for any disturbance within a specified energy or interval limit. In contrast, purely data-driven or black-box learning approaches raise concerns because proving their correctness or stability in all cases is difficult. Industry engineers thus tend to favor algorithms that come with analytic guarantees or at least are interpretable. This conservatism has motivated the hybrid approaches discussed earlier, which allow incorporation of machine learning without sacrificing the rigor of classical observers.

Another challenge is real-time computational constraints. Automotive electronic control units (ECUs) and networks have limited processing power and strict timing deadlines (often on the order of 1–10 ms for low-level control loops). Estimation algorithms with heavy computation (e.g. large neural networks or complex optimization-based observers) might be infeasible to run on current in-vehicle hardware. This drives the need for efficient implementations or hardware acceleration. It is not uncommon for advanced estimation algorithms to be prototyped in MATLAB/Simulink on a PC and then have to be simplified or optimized (e.g. using fixed-point arithmetic or reducing neural network size) for an actual vehicle ECU. The gap between academic prototypes and automotive-qualified software can be significant. Additionally, any algorithm intended for market deployment faces extensive testing: billions of miles of simulations and road tests are needed to validate autonomous vehicle software under diverse conditions. Estimation errors that are negligible on average could prove catastrophic in rare edge cases, so engineers must identify and mitigate these through scenario testing.

Environmental uncertainty presents further difficulties. Autonomous and connected vehicles operate in dynamic, unstructured environments where sensor data can degrade (heavy rain or fog affecting cameras and LiDAR, for example) and where the vehicle’s dynamics can change (tire friction changes with road surface, load variations, etc.). Estimation algorithms must be robust to these variations or adaptive enough to track them. This is closely tied to resilience: an estimator should ideally detect when sensors are providing poor data (whether due to faults, attacks, or harsh conditions) and adjust weightings or switch to fallback sensors. Industrial systems often include redundancy (e.g. multiple sensors measuring the same quantity) so that observer-based fault detection can exclude a bad sensor reading and use an alternative. For connected vehicles, redundancy can even be virtual – if a vehicle loses its own sensor, it might use a neighbor’s information via V2V communication. However, this introduces the vulnerability of connectivity: as mentioned earlier, V2V signals may arrive delayed, be missing, or be maliciously corrupted. The estimator and the overall control system must be designed to maintain safety despite these communication issues. For example, cooperative Adaptive Cruise Control (CACC) algorithms typically degrade gracefully to normal ACC if V2V data is lost or untrustworthy, relying only on local radar sensing. This graceful degradation requires observers to quickly detect anomalies in received data (e.g. a sudden improbable drop in a neighbor’s reported speed might indicate a spoofed message) and isolate or reject that data.

From an industrial standpoint, resilience against cyber-attacks is increasingly recognized as a necessary feature of autonomous vehicle software. High-profile demonstrations of car hacking have led to efforts in securing communications and ensuring observer-based monitors can promptly detect intrusions. Estimation algorithms that can reconstruct an attack signal (like the false acceleration command in a platoon) in real time are valuable for triggering mitigation strategies (such as switching to a safe mode). In this thesis, for instance, one objective is to estimate false data injection attacks on V2V links as unknown inputs and remove their effect. This kind of capability is likely to be incorporated in future vehicle intrusion detection systems (IDS) and resilient controllers.

Finally, it is important to mention the role of high-fidelity simulation and real-world testing platforms in bridging theory to practice. Researchers and engineers increasingly use platforms like CARLA (an open-source vehicle simulator) and industry tools like dSPACE/Simulink for validation of estimation algorithms in realistic scenarios. For example, in Chapter 2 of this thesis, a resilient observer for a CACC platoon under attack was tested in the CARLA simulator. The scenario involved a lead vehicle broadcasting false acceleration data to its followers; using CARLA allowed the inclusion of realistic vehicle dynamics and sensor noise. Results showed that the proposed observer could still detect and estimate the attack, though with a convergence delay of a few seconds and some noise-induced estimation error. Such studies underscore practical issues (estimator convergence time, noise sensitivity, model discrepancies) that might not be evident in pure theory. By incorporating realistic benchmarks and case studies, the development of observers can be guided to address the gaps between an idealized design and a deployable solution. In summary, while advanced estimation algorithms hold great promise for improving autonomy and safety, their industrial adoption hinges on meeting strict requirements for real-time performance, reliability, and certifiability in the unpredictable conditions of the real world.

\section{Literature Review Summary}
Before proceeding to the detailed contributions in Chapters 2–4, we summarize here the key themes from the literature that underpin this thesis. A comprehensive review was conducted on state-of-the-art observers and estimation methods relevant to resilient autonomous vehicles, spanning both classical techniques and emerging hybrid approaches.

Unknown Input and Robust Observers: Given the thesis focus on estimating states under disturbances and attacks (modeled as unknown inputs), Unknown Input Observers (UIOs) are a central tool. Classical UIO designs for linear systems require that the direct feedthrough of unknown inputs satisfies certain algebraic conditions (commonly, an input-to-output rank condition). This ensures the observer’s error dynamics can be decoupled from unknown inputs. Early works by Hou, Müller, Chen, Patton, and others formalized these conditions and developed UIOs for fault detection in the 1990s. However, these conditions are often restrictive in practice – many systems do not naturally satisfy them. To address this, researchers proposed several extensions. One approach is the use of functional observers or reduced-order observers that estimate only a function of the state (e.g. the fault itself) rather than the full state, relaxing design constraints. Another powerful approach is the incorporation of sliding-mode or high-gain techniques: Kalsi et al. (2010) showed that by adding a high-gain differentiator to generate auxiliary outputs, one can construct a sliding-mode observer even when the standard matching condition fails. Similarly, Floquet \& Barbot (2006) employed a second-order sliding mode observer to recover unknown inputs in systems not meeting UIO requirements. These methods effectively extend the class of systems for which unknown inputs can be estimated, by either transforming the system or using robust numerical differentiation to counteract unknown influences. In the realm of nonlinear systems, robust UIO theory has also advanced. For example, Zemouche \& Boutayeb (2009) developed a nonlinear observer with an $H_\infty$ performance level that can recover unknown inputs in a synchronization context. Hassan et al. (2013) designed UIOs for nonlinear time-delay systems, ensuring robustness despite the presence of delayed unknown inputs. These works and others form a rich foundation for designing observers that maintain performance under model uncertainties, disturbances, and faults – a foundation upon which our proposed methods build.

Sliding Mode and High-Gain Observers: Sliding mode observers merit special attention in the literature due to the popularity in safety-critical applications. Edwards and Spurgeon’s seminal work (1998) laid out a framework for sliding-mode observers that can exactly decouple matched disturbances. The appeal is that, once in sliding motion, the estimation error is governed by a reduced-order dynamics independent of the unknown input, yielding inherent robustness. Numerous studies have applied sliding-mode observers to vehicle systems, leveraging their finite-time convergence and robustness to parameter variations. High-order sliding modes (e.g. using super-twisting algorithms) were later introduced to improve estimation accuracy and chattering issues. As noted, Floquet et al. (2007) combined high-order sliding mode differentiators with UIO design to handle unmatched unknown inputs. In an automotive context, sliding-mode observers and differentiators have been used for estimating variables like tire forces and detecting faults/attacks. Recent literature includes applications of adaptive sliding observers to detect sensor attacks in CACC platoons, where the observer provides a real-time estimate of the attack signal. High-gain observers, while not employing discontinuous control, similarly provide strong convergence guarantees by trading off sensitivity to high-frequency noise. Meng et al. (2025) proposed a distributed high-gain observer for a network of autonomous vehicles, demonstrating how each vehicle can observe not only its own state but also a neighbor’s state with the help of fast error dynamics and V2V communication. This method falls under the broader category of distributed observers.

Distributed and Cooperative Estimation: The cooperative estimation paradigm is increasingly prominent in the literature due to the rise of connected vehicles and multi-agent robotics. Instead of designing an observer for a single system in isolation, the idea is to have a network of observers (one per vehicle/agent) that share information to improve overall estimation quality. A key benefit is tackling unobservable modes or unknown inputs that no single agent could estimate alone. Recent theoretical contributions, such as the distributed UIO by Zhao et al. (2025), show that collaborative unknown input estimation is possible in vehicle platoons: by relaxing per-vehicle observer conditions and enforcing a joint condition across the platoon, the unknown inputs (e.g. a driving disturbance or an attack affecting one vehicle) become observable to the network as a whole. Similarly, consensus-based observers and diffusion filters have been studied for sensor networks and CACC systems, where each vehicle fuses its local sensor data with neighbors’ data. Meng et al. (2025) demonstrate that a distributed high-gain observer can enhance string stability by quickly propagating state estimates along a platoon. Cooperative estimation also appears in the context of fault tolerance: if one vehicle’s sensor fails, neighboring vehicles can help estimate that vehicle’s state via inter-vehicle communication (assuming the network remains secure). The literature highlights both the potential and challenges of distributed observers – while they can significantly improve estimation under connectivity, they must be designed to handle communication delays, packet losses, and potential spoofing. Techniques like event-triggered update laws, resilient consensus (ignoring extreme outlier nodes), and decentralized attack detection are being developed to bolster cooperative observers.

Learning-Based and Adaptive Observers: In recent years, there is a surge of interest in incorporating machine learning into observers, as noted in various survey articles and emerging research. These learning-based observers encompass approaches where either part of the system model is learned from data, or the observer gains are adjusted via learning algorithms. One stream of work involves neural network observers, where a neural net is trained to emulate an inverse model or to directly output state estimates from raw sensor data. However, more relevant to control literature are neuro-adaptive observers that maintain a clear separation between a physics-based model and a learned uncertainty model. An example (already cited) is the adaptive neural observer by Jeon et al. (2024), which treats the unknown nonlinear dynamics as a function approximator tuned online with Lyapunov stability guarantees. Importantly, their design yields convergence of both the state error and the neural network weights, illustrating that learning can be done in a stable closed-loop manner. Other adaptive observer techniques include extremum-seeking observers (which adjust parameters to minimize output error), or dual-estimation approaches where a parameter estimator (which could be learning-based) runs alongside the state estimator. In the context of autonomous vehicles, such adaptive schemes have been applied to things like battery state-of-charge estimation (where neural nets learn aging effects) and vehicle dynamics estimation (learning tire force models). The hybrid model+data trend is evident: rather than rely on pure black-box models, the literature is converging on methods that treat learning as an augmentation to robust observers – aligning with the philosophy of this thesis.

Applications in Resilient Control and Attack Mitigation: The ultimate motivation behind these estimation advances is to enable resilient control strategies. A robust observer is often one half of a resilient control loop – it provides the feedback state (and possibly disturbance estimates) that a controller needs to adjust and maintain performance under adversity. For example, in a resilient vehicle platooning scenario, an observer might estimate a sudden drop in lead vehicle speed due to an attack, and the following vehicles’ controllers can then temporarily revert to a safe gap until trust is restored. Many research efforts combine observers with fault-tolerant control laws: e.g., using the estimated fault from a UIO to reconfigure control allocation (as in active fault-tolerant control systems). In the cyber-security realm, observer-based attack detection and isolation is a key enabler – by comparing output estimates with actual sensor readings (residual analysis), one can detect inconsistencies caused by attacks. Fawzi et al. (2014) and Pajic et al. (2017) studied fundamental limits on state estimation under sparse sensor attacks, showing that with enough redundancy, observers can exactly recover the true state even if some sensors are malicious. In practical terms, works like Jeon et al. (2020) on connected vehicles exemplify applying such principles: the observer not only estimates the vehicle state but also flags which sensor/communication channels may be under attack, enabling a higher-level system to exclude or downweight those signals. Cooperative Adaptive Cruise Control (CACC) testbeds often implement observers for estimating the lead vehicle’s intentions or for filtering V2V data – these ensure that a glitch or spoof in the communication does not directly corrupt the vehicle following behavior. Some literature reports using software-in-the-loop and hardware-in-the-loop experiments (with tools like MATLAB/Simulink, CARLA, or PreScan) to validate these resilient estimation and control setups before on-road trials. As a result, there is a growing body of benchmarks and case studies demonstrating that advanced observers can indeed increase the resilience of autonomous vehicles. For instance, the CARLA simulation in Chapter 2 (involving a platoon under a V2V attack) is in line with other studies that use high-fidelity simulators to test observer performance in realistic traffic scenarios. The positive outcomes in those simulations – e.g., successfully detecting an attack or maintaining safe inter-vehicle distances despite sensor faults – reinforce the practical value of the surveyed estimation techniques.

In summary, the literature shows a clear evolution toward resilient, hybrid estimation methods that are applicable to connected and autonomous vehicles. Unknown Input Observers and sliding mode observers provide a backbone for robustness against disturbances and attacks, while adaptive and learning elements address modeling mismatches and complex nonlinearities. Cooperative estimation extends these concepts to networked vehicle systems, ensuring that the benefits of connectivity (improved awareness and redundancy) can be realized without succumbing to its vulnerabilities. The subsequent chapters (Chapters 2–4) build upon these key ideas – each chapter delves into a specific advancement (from resilient UIO design to learning-enhanced observers and distributed estimation in platoons), contributing novel algorithms aligned with the directions identified in this review. Through both theoretical development and simulation validation, we aim to advance the state of the art in resilient state estimation for autonomous and connected vehicles, informed by the rich body of existing knowledge summarized above.



% \section{Background and motivation}

% Autonomous and connected vehicles rely on a tight coupling between physical dynamics, embedded computation, and networked information. In particular, cooperative driving functionalities such as Adaptive Cruise Control (ACC), Cooperative Adaptive Cruise Control (CACC), and platooning benefit from vehicle-to-vehicle (V2V) communication to improve safety and traffic efficiency. However, this connectivity comes with a structural vulnerability: decisions depend not only on local sensing, but also on communicated signals that may be delayed, missing, corrupted, or malicious.

% From an estimation viewpoint, this setting naturally leads to \emph{unknown inputs}: exogenous disturbances, modeling errors, unmeasured driver/actuator effects, and cyber-attacks that enter the dynamics or measurements. In safety-critical applications, two requirements become inseparable:
% \begin{itemize}
%     \item \textbf{Robustness and guarantees}: estimation must remain bounded and stable despite uncertainty;
%     \item \textbf{Adaptivity and accuracy}: the estimator must capture time-varying, nonlinear effects that cannot be described by a fixed nominal model.
% \end{itemize}

% Purely data-driven estimators can be accurate but may be difficult to certify. Conversely, purely model-based observers provide guarantees but may be conservative or fail when classical existence conditions are violated (e.g., restrictive rank/matching constraints for unknown input observers). This thesis adopts a hybrid perspective: combine physics-informed observer design with learning and trust mechanisms so that estimation remains reliable under uncertainty, cyber threats, and distributed information exchange.

% \section{Problem statement and objectives}

% The general problem addressed in this thesis is \emph{resilient state estimation} for connected vehicles, where disturbances and attacks are modeled as unknown inputs and where communication can be unreliable or adversarial. The targeted application layer covers CACC/platooning scenarios in which estimating states and reconstructing unknown signals is a prerequisite for maintaining safety and performance.

% More specifically, the objectives are to:
% \begin{itemize}
%     \item Design observer-based estimators that can simultaneously track vehicle states and reconstruct unknown inputs affecting the closed-loop behavior.
%     \item Address cyber-attacks on V2V signals (notably false data injection and delay-induced corruption) by treating them as unknown inputs and reconstructing them in real time.
%     \item Extend estimation from a single-vehicle viewpoint to \emph{cooperative} settings, where each agent fuses local sensing with exchanged information.
%     \item Integrate learning mechanisms to approximate nonlinear, unmodeled dynamics while preserving stability guarantees.
%     \item Validate the proposed methods through simulation studies under nominal conditions and under adverse perturbations (disturbances, communication issues, and attacks).
% \end{itemize}

% The central hypothesis is that \emph{hybrid observer architectures} can achieve both: (i) rigorous error bounds and stability properties inherited from control theory, and (ii) improved estimation accuracy by learning structured uncertainty from data, while remaining compatible with the constraints of connected vehicle systems.

\section{Positioning and challenges}

Three methodological challenges motivate the developments in this manuscript.
\begin{itemize}
    \item \textbf{Unknown input reconstruction under realistic assumptions:} estimating attacks/disturbances is not only a detection problem; reconstruction is needed for mitigation and control. Yet, classical unknown input observer (UIO) designs rely on rank/matching conditions that may fail for practical vehicle models.
    \item \textbf{Nonlinear and time-varying uncertainty:} vehicle dynamics include effects that are difficult to capture with fixed parametric models (e.g., aggregated tire/road effects, drag, actuator dynamics variations). Learning can approximate these uncertainties, but must be embedded into a stable observer structure.
    \item \textbf{Distributed estimation with adversarial information:} in platoons, estimation depends on exchanged data. Robustness requires mechanisms to reduce the influence of compromised or unreliable neighbors without destroying convergence.
\end{itemize}

The remainder of the thesis proposes observer designs addressing these points in a unified resilient estimation viewpoint.

\section{Approach overview}

The thesis follows a consistent workflow: (i) formulate vehicle/communication models with disturbances and attacks as unknown inputs, (ii) design observers with explicit robustness properties, (iii) enrich the estimators with learning or trust layers when needed, and (iv) validate by simulation on representative scenarios.

\begin{figure}
    \centering
    \includegraphics[width=0.9\linewidth]{ch1/fig/hybrid_model_2.png}
    \caption{Hybrid resilient estimation framework for connected vehicles}
    \label{fig:hybrid_model}
\end{figure}

\begin{figure}
    \centering
    \includegraphics[width=0.9\linewidth]{ch1/fig/under_attack.png}
    \caption{CACC under cyber-attack scenario}
    \label{fig:under_attack}
\end{figure}

\begin{figure}
    \centering
    \includegraphics[width=0.9\linewidth]{ch1/fig/Structure_these_2.png}
    \caption{Organization of the thesis}
    \label{fig:structure_these}
\end{figure}

\section{Main contributions}

The main contributions of this thesis can be summarized as follows:
\begin{itemize}
    \item \textbf{Resilient estimation and attack reconstruction for CACC:} an observer-based framework that estimates vehicle states and reconstructs corrupted communicated signals modeled as unknown inputs.
    \item \textbf{Learning with generalized UIOs for nonlinear estimation:} a hybrid teacher--student architecture combining a generalized UIO (robust bounded estimation) with a neural adaptive observer (learning structured uncertainty) under LMI-based stability conditions.
    \item \textbf{Trust-aware distributed state estimation for platoons:} a distributed observer architecture that adapts neighbor influence using a trust/divergence metric to maintain estimation performance under cyber/communication attacks.
\end{itemize}

Overall, the thesis contributes observer designs that combine robustness, adaptivity, and resilience in connected vehicle applications.

\section{Organization of the manuscript}

The manuscript is organized as follows.
\begin{itemize}
    \item Chapter~\ref{chp3_chap} studies resilient state estimation and cyber-attack reconstruction for connected autonomous vehicles in a CACC setting.
    \item Chapter~\ref{chp4_chap} introduces a learning-based unknown input observer framework for robust nonlinear estimation, combining a generalized UIO with a neural adaptive observer.
    \item Chapter~\ref{chp5_chap} proposes a resilient trust-aware distributed observer design for connected vehicle platoons under adversarial conditions.
    \item The conclusion summarizes the main results and outlines perspectives for future work.
\end{itemize}
